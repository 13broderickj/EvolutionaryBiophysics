\documentclass{article}

% Package definitions (add as necessary):
\usepackage{fancyhdr} % Required for header
\usepackage{extramarks} % Required for header
\usepackage{amsmath}

% Formatting margins and spacings:
\topmargin=-0.45in
\evensidemargin=0in
\oddsidemargin=0in
\textwidth=6.5in
\textheight=9.0in
\headsep=0.25in
\setlength\parindent{0pt}

% Header formatting:
\pagestyle{fancy}
\fancyhf{}
\lhead{Nicholas Barendregt} % Author name
\chead{PHYS 320} % Class name
\rhead{Final Project: Mutation Notes} % Assignment name
\rfoot{\thepage}

% Assignment begins here:

\begin{document}
	\section{Developing the Model}
		Consider a single base pair site on a strand of DNA. The allowed states of this site make up a finite state space $E = \{A, G, C, T\}$. Imagine a finite-state Markov process as a model for the mutation of this site: the site starts in a given state $X_0$, and in each time step $\Delta t$, there is a well-defined probability of transition to state $X_i \in E$. 
		
		Given this physical model, we can construct the following transition matrix:
		
		\begin{equation}
			\Omega = \begin{pmatrix}
				-\mu_A & \mu_{GA} & \mu_{CA} & \mu_{TA} \\
				\mu_{AG} & -\mu_G & \mu_{CG} & \mu_{TG} \\
				\mu_{AC} & \mu_{GC} & -\mu_G & \mu_{TG} \\
				\mu_{AT} & \mu_{GT} & \mu_{CT} & -\mu_T
			\end{pmatrix}.
			\label{eq:omega_matrix_general}
		\end{equation}
		In \eqref{eq:omega_matrix_general}, we enforce that $\mu_ii$ is set so the columns of $\Omega$ sum to one. For our basic model, we will assume that all transitions are equally likely. This allows us to reduce $\Omega$ into the simple form 
		
		\begin{equation}
		\Omega = \begin{pmatrix}
			-\frac{3}{4} & \frac{1}{4} & \frac{1}{4} & \frac{1}{4} \\
			\frac{1}{4} & -\frac{3}{4} & \frac{1}{4} & \frac{1}{4} \\
			\frac{1}{4} & \frac{1}{4} & -\frac{3}{4} & \frac{1}{4} \\
			\frac{1}{4} & \frac{1}{4} & \frac{1}{4} & -\frac{3}{4}
		\end{pmatrix}.
		\label{eq:omega_matrix}
		\end{equation}
		
		Now that we have the transition matrix \eqref{eq:omega_matrix}, we can begin to construct our model. Consider the probability vector $\mathbf{P}(t) = \left[p_A(t)\ p_G(t)\ p_C(t)\ p_T(t)\right]^T$. We know from studying the master equation,
		
		\begin{equation}
			p_i(t+\Delta t) = p_i(t) - p_i(t)\mu_ii\Delta t +\sum_{j\neq i}p_j(t)\mu_{ji}\Delta t,
			\label{eq:master_equation_one_base}
		\end{equation}
		because probability must be conserved. We can expand this notion to get an expression for $\mathbf{P}(t+\Delta t)$:
		
		\begin{equation}
			\mathbf{P}(t+\Delta t) = \mathbf{P}(t) + \Omega\mathbf{P}(t)\Delta t.
			\label{eq:master_equation_P}
		\end{equation}
		
		We can see that equation \eqref{eq:master_equation_P} just represents the system of all equations \eqref{eq:master_equation_one_base}. Dividing both sides of \eqref{eq:master_equation_P} by $\Delta t$ and taking the limit as $\Delta t$ goes to zero gives us the differential equation
		
		\begin{equation}
			\frac{d\mathbf{P}(t)}{dt} = \Omega\mathbf{P}(t).
			\label{eq:P_ode}
		\end{equation}
		
		Equation \eqref{eq:P_ode} can be simply solved by direct integration, giving the solution
		
		\begin{equation}
			\mathbf{P}(t) = \mathbf{P}(0)e^{\Omega t},
			\label{eq:P_solution}
		\end{equation}
		
		where $e^\Omega t$ is given by the definition of a matrix exponential represented by a Taylor series,
		
		\begin{equation}
			e^{\Omega t} = \sum_{k=0}^{\infty}\Omega^k\frac{t^k}{k!}.
			\label{eq:matrix_exponential}
		\end{equation}
		
	\section{Application of the Model: Deriving Long-Run Similarity}
		
		Consider $P(t) = e^\Omega t$. Using $\Omega$ from \eqref{eq:omega_matrix} and \eqref{eq:matrix_exponential} gives the following form for $P(t)$:
		
		\begin{equation}
			P(t) = \begin{pmatrix}
				\frac{1}{4} + \frac{3}{4}e^{-t} & \frac{1}{4} - \frac{1}{4}e^{-t} & \frac{1}{4} - \frac{1}{4}e^{-t} & \frac{1}{4} - \frac{1}{4}e^{-t} \\
				\frac{1}{4} - \frac{1}{4}e^{-t} & \frac{1}{4} + \frac{3}{4}e^{-t} & \frac{1}{4} - \frac{1}{4}e^{-t} & \frac{1}{4} - \frac{1}{4}e^{-t} \\
				\frac{1}{4} - \frac{1}{4}e^{-t} & \frac{1}{4} - \frac{1}{4}e^{-t} & \frac{1}{4} + \frac{3}{4}e^{-t} & \frac{1}{4} - \frac{1}{4}e^{-t} \\
				\frac{1}{4} - \frac{1}{4}e^{-t} & \frac{1}{4} - \frac{1}{4}e^{-t} & \frac{1}{4} - \frac{1}{4}e^{-t} & \frac{1}{4} + \frac{3}{4}e^{-t}
			\end{pmatrix}
			\label{eq:P_application}
		\end{equation}
		Plugging this result into equation \eqref{eq:P_solution} and taking the limit as \textit{t} goes to infinity gives the long-run expected probability distribution,
		
		\begin{equation}
			\lim\limits_{t\rightarrow\infty}\mathbf{P}(t) = \frac{1}{4}\mathbf{P}(0).
			\label{eq:long-run_P}
		\end{equation}
		
		We can interpret \eqref{eq:long-run_P} as follows: our DNA site starts at an initial probability distribution $\mathbf{P}(0)$. As time progresses, the probability that this site follows the initial distribution decreases by a factor of four. This means that if we consider not one site, but the entire strand of DNA, the probability that after a long time the mutated strand is identical to the starting strand is one fourth. 
\end{document}
